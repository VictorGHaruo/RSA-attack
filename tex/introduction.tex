\section{Introdução}

O presente trabalho consolida os fundamentos teóricos da disciplina de Álgebra e Criptografia através de uma análise prática da segurança do sistema RSA. Embora matematicamente robusto, o RSA torna-se vulnerável quando parâmetros são escolhidos de forma inadequada. Nesse contexto, este estudo examina a eficiência e a complexidade de implementação de diferentes vetores de ataque explorando falhas estruturais e de configuração. A análise comparativa abrange desde métodos genéricos, como a Força Bruta e a Fatoração de Fermat, até técnicas específicas baseadas em propriedades algébricas, incluindo o Ataque de Módulo Comum, os Algoritmos de Pollard, o Ataque ao Pequeno Expoente Público e o Ataque de Wiener.
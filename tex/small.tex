\section{Ataque da Pequena Chave Pública}

Como em todo processo, sempre buscamos maior rapidez e eficiência, assim,
podemos ficar bastante tentados a possuir um expoente público pequeno, 
para que, no processo de encriptação $c \equiv m^{e} \pmod{n}$, a demora 
computacional seja reduzida.

De fato, parece tentador e seguro, já que a \textit{dificuldade} de quebra
do RSA se deve à fatoração de $n$, então diminuir o expoente parece
razoável.

Entretanto, existe uma limitação nesse processo, uma vez que, caso $m^e < n$,
a criptografia $c \equiv m^e \pmod{n} \Rightarrow c = m^e$, então basta 
$$m = \sqrt[e]{c}$$ para recuperar a mensagem e novamente \textbf{sem 
precisar fatorar n.}

\subsection{O Ataque de Hastad}

Suponhamos que um emissor deseja enviar uma mensagem encriptada
$M$ para os receptores $R_1, R_2, \cdots,R_k$. Cada um dos receptores tem
a sua chave pública $(N_i, e_i)$. Vamos assumir que $M$ é menor que
qualquer dos $N_i$. Um intruso pode interceptar a ligação sem que
o emissor perceba e coletar os $k$ criptogramas.

Simplificando, considere que todos possuem $e_i = 3$. Podemos mostrar que se
$k \geq 3$, o intruso consegue recuperar a mensagem $M$ a partir de $C_1, C_2, C_3, \cdots$
criptografia da mensagem em cada emissão.

Pois veja, se temos:
$$
\begin{cases}
    C_1 \equiv m^3 \pmod{N_1} \\ 
    C_2 \equiv m^3 \pmod{N_2} \\
    C_3 \equiv m^3 \pmod{N_3}
\end{cases}
$$
Temos também que $\gcd(N_i, N_j) = 1 \quad \forall i,j$ (Ou seja, coprimos, pois caso não fossem,
poderíamos encontrar a fatoração deles). E pelo Teorema do Resto Chinês (TRC),
podemos encontrar $C'$ tal que $C' \equiv m^3 \pmod{N_1N_2N_3}$.

Assim, como 
$$
\begin{cases}
    m < N_1 \\
    m < N_2 \\
    m < N_3
\end{cases} \Rightarrow m^3 < N_1N_2N_3
$$
Portanto, $C' = m^3 \Rightarrow m = \sqrt[3]{C'}$.

De forma mais geral, se os expoentes de encriptação forem todos iguais a $e$,
pode-se recuperar $M$ assim que se tenham $k > e$ onde $k$ é o número de
criptogramas interceptados. Portanto, este ataque só será bem sucedido
se o expoente público $e$ for relativamente pequeno.

\subsection{Conclusão}

É essencial, mesmo que facilite a encriptação, o uso de expoentes relativamente
altos. Uma boa recomendação é o uso de no mínimo $2^{16}+1$, que, de certa forma,
facilita esses cálculos e ainda é relativamente grande. 
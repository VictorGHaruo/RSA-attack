\section{Fatoração de Fermat}

Em particular, suponha que haja o conhecimento prévio de que a fatoração da chave pública sejam números próximos de $\sqrt{n}$, mostraremos um método eficiente desenvolvido pelo matemático francês Pierre de Fermat (1601-1665) capaz de descobrir $p$ e $q$ com voracidade.

\subsection{Método}
Seja $n$ um número inteiro maior do que 1 que desejamos fatorar. Note que, o objetivo principal do método é encontrar inteiros não negativos $x$ e $y$, tais que $n = x^{2} - y^{2}$; já que é conhecida a identidade $x^{2} - y^{2} = (x-y)(x+y)$, daí encontramos uma fatoração de $n$ que não necessariamente é em fatores primos. Entretanto, como no sistema RSA trabalhamos com semiprimos, o método encontra uma fatoração em primos.

\subsubsection{Algoritmo de Fatoração}

\begin{definicao}
Seja $n$ um inteiro ímpar maior do que $1$. \\
    \textbf{\underline{Passo 1:}} Calcule $\sqrt{n}$
    \begin{itemize}
        \item Se $\sqrt{n}$ for um número inteiro, então o processo terminou, pois n é um quadrado perfeito e basta, então, tomarmos $x=\sqrt{n}$ e $y-0$.
        \item Se $\sqrt{n}$ não for um número inteiro, defina $\lfloor \sqrt{n} \rfloor$ e siga para o \textbf{Passo 2}
    \end{itemize}
    \textbf{\underline{Passo 2:}} Faça $x=b+1$. 
    \begin{itemize}
        \item Se $x = \frac{n+1}{2}$, então $n$ é primo, $y = \sqrt{x ^{2}- n}$ e o processo terminou. Nesse caso, observe que 
        
        \begin{align*}
            y &= \sqrt{x^{2} - n} \\
            &= \sqrt{\left(\frac{n+1}{2}\right)^{2} - n} \\
            &= \frac{n - 1}{2}.
        \end{align*}
        \item Se $x \ne \frac{n+1}{2}$, então siga para o \textbf{Passo 3}
    \end{itemize}
    \textbf{\underline{Passo 3:}} Calcule $y = \sqrt{x^{2} - n}$.

    \begin{itemize}
        \item Se $y$ for um número inteiro, então o processo terminou pois n é composto e obtivemos inteiros não negativos $x$ e $y$ tais que $x>y$ e $n = x^{2} - y^{2}$.
        \item Se y não for um nuúmero inteiro, desconsidere o valor anterior de b, defina $b=x$ e volte para o \textbf{Passo 2}, redefinindo x conforme este novo valor de $b$.
    \end{itemize}
\end{definicao}

\subsubsection{Por que o método funciona?}

Na demonstração consideramos que $n$ é um número ímpar, pois, caso contrário seria possível escrever $n = 2^{a} \cdot b$ para $a$ inteiro positivo e algum $b$ inteiro positivo ímpar. Dado que $2$ é primo, bastaria fatorar $b$ para descobrir a fatoração de $n$ como produto de números primos, logo, recaindo no problema de se fatorar números ímpares.

\subsection{Implementação Computacional}

\begin{lstlisting}[style=PythonStyle, caption=Fermat Algorithm, label=code:Fermat]
def fermat(N):
    x = math.isqrt(N)
    
    if (N%2 == 0):
        return (2, N//2)

    if (x*x == N):
        return (x, x)

    while x != (N + 1)//2:
        x = x + 1
        w = pow(x, 2) - N
        y = math.isqrt(w)
        if (y*y  == w):
            return (x-y, x+y)

    return (1, N)
\end{lstlisting}

\subsubsection{Simulação}
Como comentado no início da seção, a Fatoração de Fermat é um mecânismo útil quando os fatores do número são próximos, entretanto, vamos analisar o quão melhor o algoritmo se torna ao tomar o pior e melhor caso; isto é, no pior caso tomar um semiprimo $n = 3 \cdot p$ - tomamos 3 pois o algoritmo trata o 2 como um caso particular e por isso não segue diretamente o algoritmo -  e o melhor caso em que $p$ e $q$ são os primos mais próximos. Além disso, incrementamos os digitos do número e vemos o tempo decorrido para calcular a decomposiççao em primos.


\begin{table}[H]
\centering
\caption{Tabela para Fatoração de Semiprimos}
\label{tab:fatores}
\begin{tabular}{r l}
\toprule
Semiprimo (N) & Fatores (p, q) \\
\midrule
15 & 3\cdot 5\\
143 & 11\cdot 13\\
2491 & 47\cdot 53\\
47053 & 211\cdot 223\\
304679 & 547\cdot 557\\
5494327 & 2341\cdot 2347\\
76562491 & 8747\cdot 8753\\
816359183 & 28571\cdot 28573\\
7844290183 & 88547\cdot 88589\\
83274953467 & 288571\cdot 288577\\
150986644891 & 388567\cdot 388573\\
\bottomrule
\end{tabular}
\end{table}

\begin{table}[H]
\centering
\caption{Tabela para Fatoração de Semiprimos}
\label{tab:fatores_f}
\begin{tabular}{r l} 
\toprule
Semiprimo (N) & Fatores (p, q) \\
\midrule
93 & $3 \cdot 31$\\
267 & $3 \cdot 89$\\
1569 & $3 \cdot 523$\\
19473 & $3 \cdot 6491$\\
499461 & $3 \cdot 166487$\\
2899473 & $3 \cdot 966491$\\
17899473 & $3 \cdot 5966491$\\
107899431 & $3 \cdot 35966477$\\
1607899443 & $3 \cdot 535966481$\\
13607899449 & $3 \cdot 4535966483$\\
874607899119 & $3 \cdot 291535966373$\\
\bottomrule
\end{tabular}
\end{table}

\begin{figure}[H]
    \centering
    \begin{minipage}{0.49\textwidth}
        \centering
        \includegraphics[width=\linewidth]{img/nologFermat.png}
        \caption{Sem Escala Logarítmica.}
        \label{fig:a}
    \end{minipage}
    \hfill
    \begin{minipage}{0.49\textwidth}
        \centering
        \includegraphics[width=\linewidth]{img/logFermat.png}
        \caption{Escala Logarítmica.}
        \label{fig:b}
    \end{minipage}
\end{figure}


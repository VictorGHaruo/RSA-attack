\documentclass{article}
\usepackage[utf8]{inputenc}
\usepackage[brazil]{babel}
\usepackage{amsmath}
\usepackage{amssymb}
\usepackage{amsfonts}
\usepackage{graphicx}
\usepackage{float}
\usepackage[shortlabels]{enumitem}
\usepackage{fancyhdr} 
\usepackage{lastpage} 
\usepackage{amsthm}
\usepackage{multicol}
\usepackage{booktabs}
\usepackage{pgfplots}
\usepackage{tikz}
\usepackage{wrapfig}
\usepackage{subcaption}
\usepackage[colorlinks=true, linkcolor=black, urlcolor=blue]{hyperref}
\usepackage{xurl}
\usepackage[alf]{abntex2cite} 
\usepackage{listings}
\usepackage{xcolor}
\usepackage{mdframed} 
\pgfplotsset{compat=1.18}

%--------Muda as Cores do Template do Código--------
\definecolor{bggray}{RGB}{245, 245, 245}     
\definecolor{keywordblue}{RGB}{0, 0, 180}    
\definecolor{stringred}{RGB}{180, 0, 0}      
\definecolor{commentgreen}{RGB}{34, 139, 34} 
\definecolor{numberpurple}{RGB}{128, 0, 128} 

\lstdefinestyle{PythonStyle}{
    backgroundcolor=\color{bggray},   
    basicstyle=\ttfamily\footnotesize, 
    frame=single,                     
    rulecolor=\color{gray},           
    keywordstyle=\color{keywordblue}\bfseries, 
    stringstyle=\color{stringred},    
    commentstyle=\color{commentgreen}\itshape,
    numberstyle=\color{numberpurple}, 
    showstringspaces=false,           
    numbers=left,                     
    numbersep=5pt,                    
    tabsize=4,                        
    captionpos=b,                     
    breaklines=true,                      language=Python                   
}
%------------------------------------------------

%--------Configuração do Header e Footer--------
\pagestyle{fancy}
\fancyhf{} 

\lhead{\includegraphics[height=0.29cm]{img/emap.png}} 
\rhead{\textbf{Fundação Getúlio Vargas}}

\lfoot{Álgebra \& Criptografia}
\rfoot{Página \thepage\ de \pageref{LastPage}} 

\renewcommand{\headrulewidth}{0.4pt}
\renewcommand{\footrulewidth}{0.4pt}
%-----------------------------------------------

%-------------------Caixinha--------------------
\newtheorem{definicao_interna}{Definição} 
\newtheorem{teorema}{Teorema}[section]

\newmdenv[
  linecolor=black,
  linewidth=1pt,
  roundcorner=5pt,
  backgroundcolor=gray!10,
  innertopmargin=5pt,
  innerbottommargin=10pt,
  innerleftmargin=10pt,
  innerrightmargin=10pt,
]{definicao}
%----------------------------------------------


\begin{document}
    \begin{titlepage}
    \centering 


    \includegraphics[width=0.2\textwidth]{img/fgv.png} 
    
    \vspace{0.3cm} 

    {\Large FUNDAÇÃO GETULIO VARGAS \\}
    {\normalsize ESCOLA DE MATEMÁTICA APLICADA \\}
    {\normalsize Álgebra \& Criptografia}

    \vfill 

    {\bfseries\Large Ataque ao RSA }

    \vfill 

    \begin{tabular}{c}
        ERIC MANOEL RIBEIRO DE SOUSA\\
        LUAN RODRIGUES DE CARVALHO\\
        RODRIGO SEVERO ARAÚJO\\
        VICTOR GABRIEL HARUO IWAMOTO \\
    \end{tabular}

    \vspace{2.cm} 

    {Rio de Janeiro -- RJ \\} 
    {Outubro 2025}

\end{titlepage}
    \section{Introdução}

O presente trabalho consolida os fundamentos teóricos da disciplina de Álgebra e Criptografia através de uma análise prática da segurança do sistema RSA. Embora matematicamente robusto, o RSA torna-se vulnerável quando parâmetros são escolhidos de forma inadequada. Nesse contexto, este estudo examina a eficiência e a complexidade de implementação de diferentes vetores de ataque explorando falhas estruturais e de configuração. A análise comparativa abrange desde métodos genéricos, como a Força Bruta e a Fatoração de Fermat, até técnicas específicas baseadas em propriedades algébricas, incluindo o Ataque de Módulo Comum, os Algoritmos de Pollard, o Ataque ao Pequeno Expoente Público e o Ataque de Wiener.
    \section{Fatoração (Naive Algorithm)}
Uma abordagem natural contra o sistema criptogáfico RSA consiste na fatoração da chave pública $n$ (um semiprimo) com fim de encontrar os fatores primos $p$ e $q$. Por conseguinte, descobre-se a classe de congruência $\phi(n) = (p-1)(q-1)$ que daí basta determinar $d$ -  inverso multiplicativo de $e$ - resultando na completa quebra da criptografia e, portanto, a exposição da mensagem.

Apesar disso, a fatoração de números inteiros é um problema computacionalmente custoso visto que hodiernamente o sistema RSA adota como padrão mínimo 1024 bits como o tamanho da chave pública, equivalente a cerca de 300 dígitos. Ademais, os algoritmos tendem a ter maior dificuldade para encontrar números semiprimos como no caso do RSA, daí com as tecnologias disponíveis em estimativa demora de 10 à 15 anos de computação para descriptografar com tal método. 

Logo, a fatoração de números inteiras é virtualmente inofensivo e inviável para um ataque sério ao RSA, entretanto, é útil no âmbito acadêmico dado que é uma ótima base de comparação com os demais ataques.

\subsection{Implementação Computacional}


O algoritmo de "força bruta" para fatoração é conhecido como Divisão por Tentativa (Trial Division). Embora sua premissa seja simples, é possível otimizá-lo para contornar casos desnecessários como:

\begin{enumerate}
    \item \textbf{Limite de $\sqrt{n}$:} Basta testar divisores até a raiz quadrada de $n$. Se $n$ possuir um fator $a > \sqrt{n}$, ele obrigatoriamente terá um fator $b < \sqrt{n}$ (tal que $n = ab$), que já teria sido encontrado.
    
    \item \textbf{Tratamento do fator 2:} O número 2 é o único primo par. Ele pode ser tratado em um loop separado, o que permite que o loop principal teste apenas divisores ímpares.
    
    \item \textbf{Teste de ímpares:} Após remover todos os fatores 2, o $n$ restante é ímpar. Seus fatores primos também serão ímpares. Portanto, o loop principal pode testar apenas divisores a partir de 3, incrementando o divisor de 2 em 2 (3, 5, 7, ...).
\end{enumerate}

\begin{lstlisting}[style=PythonStyle, caption=Naive Algorithm, label=code:Naive]
    def naive(n):
        begin = time.time()
        div = []
        while (n%2 == 0):
            div.append(2)
            n = n//2
        i = 3
        while i*i <= n:
            print(i)            
            while n%i == 0:
                div.append(i)
                n = n//i
            i = i + 2

        end = time.time()
        if n > 1:
            div.append(n)
        return div, end-begin

\end{lstlisting}

\subsection{Simulação Computacional}

Com o intuito de compreender a natureza do método, realizou-se testes computacionais focados no pior caso do algoritmo: a fatoração de semiprimos. O experimento consistiu em variar o número de dígitos do semiprimo e comensurar o tempo de execução decorrido.

\begin{table}[H]
    \centering
    \label{tab:tempos_sample}
    \begin{tabular}{cc}
\toprule
Digitos & Tempo (s) \\
\midrule
1 & $2.38 \times 10^{-6}$ \\
6 & $5.69 \times 10^{-4}$ \\
11 & 0.30 \\
16 & 123.84 \\
\bottomrule
\end{tabular}
\caption{Amostragem dos Tempos de Fatoração (a cada 5 itens)}
\end{table}

\begin{table}[H]
\centering
\caption{Semiprimos e seu respectivo número de dígitos}
\label{tab:semiprimes_digitos}
\begin{tabular}{rc}
\toprule
\textbf{Semiprimo (n)} & \textbf{N° de Dígitos} \\
\midrule
6 & 1 \\
77 & 2 \\
989 & 3 \\
2291 & 4 \\
97627 & 5 \\
358091 & 6 \\
8846573 & 7 \\
63451711 & 8 \\
553789213 & 9 \\
5276275391 & 10 \\
48965927779 & 11 \\
868082737663 & 12 \\
5163693436199 & 13 \\
53684551531801 & 14 \\
635621477042171 & 15 \\
6750421608780299 & 16 \\
68569780649272979 & 17 \\
\bottomrule
\end{tabular}
\end{table}

Por meio dos dados presentes na tabela, é plausível julgar um crescimento exponencial. Essa natureza é visualmente confirmada nos gráficos da Figura \ref{fig:asew}.

O gráfico da esquerda (Figura \ref{fig:asew}) plota os dados em escala linear; a curva explode de tal forma que os primeiros pontos se tornam indistinguíveis, um comportamento clássico de crescimento exponencial.

O gráfico da direita (Figura \ref{fig:tfdras}), por sua vez, aplica uma escala logarítmica ao eixo Y (Tempo). Como esperado de uma função exponencial, os pontos se alinham em uma reta, confirmando a relação $Tempo \approx e^k$, onde $k$ é o número de dígitos.

\begin{figure}[H]
    \centering
    \begin{minipage}{0.49\textwidth}
        \centering
        \includegraphics[width=\linewidth]{img/nolog.png}
        \caption{Sem Escala Logarítmica.}
        \label{fig:asew}
    \end{minipage}
    \hfill
    \begin{minipage}{0.49\textwidth}
        \centering
        \includegraphics[width=\linewidth]{img/log.png}
        \caption{Escala Logarítmica.}
        \label{fig:tfdras}
    \end{minipage}
\end{figure}

O resultado final é categórico: o algoritmo levou 287 segundos (mais de 4 minutos e meio) para fatorar um semiprimo de apenas 17 dígitos. Considerando que chaves RSA (que são semiprimos) utilizavam como padrão mínimo 1024 bits (cerca de 300 dígitos), fica evidente que o método de divisão por tentativa é computacionalmente inviável para qualquer aplicação criptográfica real.



    \section{Fatoração de Fermat}

Em particular, suponha que haja o conhecimento prévio de que a fatoração da chave pública sejam números próximos de $\sqrt{n}$, mostraremos um método eficiente desenvolvido pelo matemático francês Pierre de Fermat (1601-1665) capaz de descobrir $p$ e $q$ com voracidade.

\subsection{Método}
Seja $n$ um número inteiro maior do que 1 que desejamos fatorar. Note que, o objetivo principal do método é encontrar inteiros não negativos $x$ e $y$, tais que $n = x^{2} - y^{2}$; já que é conhecida a identidade $x^{2} - y^{2} = (x-y)(x+y)$, daí encontramos uma fatoração de $n$ que não necessariamente é em fatores primos. Entretanto, como no sistema RSA trabalhamos com semiprimos, o método encontra uma fatoração em primos.

\subsubsection{Algoritmo de Fatoração}

\begin{definicao}
Seja $n$ um inteiro ímpar maior do que $1$. \\
    \textbf{\underline{Passo 1:}} Calcule $\sqrt{n}$
    \begin{itemize}
        \item Se $\sqrt{n}$ for um número inteiro, então o processo terminou, pois n é um quadrado perfeito e basta, então, tomarmos $x=\sqrt{n}$ e $y=0$.
        \item Se $\sqrt{n}$ não for um número inteiro, defina $\lfloor \sqrt{n} \rfloor$ e siga para o \textbf{Passo 2}
    \end{itemize}
    \textbf{\underline{Passo 2:}} Faça $x=b+1$. 
    \begin{itemize}
        \item Se $x = \frac{n+1}{2}$, então $n$ é primo, $y = \sqrt{x ^{2}- n}$ e o processo terminou. Nesse caso, observe que 
        
        \begin{align*}
            y &= \sqrt{x^{2} - n} \\
            &= \sqrt{\left(\frac{n+1}{2}\right)^{2} - n} \\
            &= \frac{n - 1}{2}.
        \end{align*}
        \item Se $x \ne \frac{n+1}{2}$, então siga para o \textbf{Passo 3}
    \end{itemize}
    \textbf{\underline{Passo 3:}} Calcule $y = \sqrt{x^{2} - n}$.

    \begin{itemize}
        \item Se $y$ for um número inteiro, então o processo terminou pois n é composto e obtivemos inteiros não negativos $x$ e $y$ tais que $x>y$ e $n = x^{2} - y^{2}$.
        \item Se y não for um número inteiro, desconsidere o valor anterior de b, defina $b=x$ e volte para o \textbf{Passo 2}, redefinindo x conforme este novo valor de $b$.
    \end{itemize}
\end{definicao}

\subsubsection{Por que o método funciona?}

Para demonstrar o funcionamento do algoritmo, dividiremos a prova em passos, mostrando o funcionamento para números compostos que não são quadrados perfeitos e provando que o algoritmo é finito.

Na demonstração, consideramos que $n$ é um número ímpar. Caso contrário, seria possível escrever $n = 2^{a} \cdot b$, com $a$ inteiro positivo e $b$ inteiro positivo ímpar. Dado que $2$ é primo, bastaria fatorar $b$ para descobrir a fatoração de $n$ em números primos, recaindo assim no problema de fatorar números ímpares.

Dessa maneira, vamos mostrar que, para o caso em que $n$ é inteiro composto, ímpar e maior do que 1, sempre encontraremos o valor de $x$.

Seja $n$ um inteiro composto, ímpar, maior do que 1 e que não seja um quadrado perfeito. Seja $n=a \cdot b$ uma fatoração para $n$, com $a$ e $b$ inteiros tais que $1 < a < b < n$. Note que $a \neq b$, já que estamos supondo que $n$ não é quadrado perfeito; além disso, $a$ e $b$ devem ser ímpares.

Devemos assegurar a existência de números inteiros positivos $x$ e $y$ tais que $n=x^2 - y^2$.

\[
\begin{cases}
    n = a \cdot b \\
    n = x^2 - y^2
\end{cases}
\]

Assim, temos que $a \cdot b = (x-y)(x+y)$. Tomemos $a = x-y$ e $b = x+y$. Logo:

\[
\begin{cases}
    x = \frac{a+b}{2}\\
    y = \frac{b-a}{2}
\end{cases}
\]

Verifiquemos que os valores de $x$ e $y$ satisfazem todas as condições necessárias:

\begin{enumerate}
    \item Como $1 < a < b < n$, então $a+b > 0$ e $b-a > 0$. Assim, $x$ e $y$ são positivos.
    
    \item Como $a$ e $b$ são ímpares, a soma $a+b$ e a diferença $b-a$ são pares. Logo, $x$ e $y$ são inteiros.
    
    \item Note que:
    \[
    x^2 - y^2 = \left(\frac{a+b}{2}\right)^2 - \left(\frac{b-a}{2}\right)^2 = \frac{(a+b)^2 - (b-a)^2}{4} = \frac{4ab}{4} = a \cdot b = n
    \]
    Daí, $n = x^2 - y^2$, o que implica $y = \sqrt{x^2 - n}$, dado que $y$ é positivo.

    \item Sabemos que, para $a \neq b$, $(\sqrt{a} - \sqrt{b})^2 > 0$. Assim, $a - 2\sqrt{ab} + b > 0$, o que implica $\sqrt{ab} < \frac{a+b}{2}$.
    Dessa forma, $\sqrt{n} < x$. Portanto, $x > \lfloor \sqrt{n} \rfloor$. Como $x$ é inteiro, temos que $\lfloor \sqrt{n} \rfloor + 1 \le x$.

    \item Sendo $b > a \geq 2$ (pois $n$ é composto ímpar, logo os fatores são $\ge 3$), temos:
    \begin{itemize}
        \item $2 < b \implies a = \frac{a}{2} \cdot 2 < \frac{a}{2} \cdot b = \frac{ab}{2}$. Logo, $a < \frac{ab}{2}$.
        \item De $2 \leq a$, segue que $b = 2 \cdot \frac{b}{2} \leq a \cdot \frac{b}{2} = \frac{ab}{2}$. Logo, $b \leq \frac{ab}{2}$.
    \end{itemize}
    Somando as desigualdades:
    \[
    a + b < \frac{ab}{2} + \frac{ab}{2} = ab < ab + 1
    \]
    Dividindo por 2:
    \[
    \frac{a + b}{2} < \frac{ab + 1}{2} = \frac{n + 1}{2}
    \]
    Portanto, {$x < \frac{n + 1}{2}$}.

    \vspace{0.5cm}

    \item Note que $x - y = \frac{a + b - (b - a)}{2} = \frac{2a}{2} = a$.
    Como $n$ é composto, temos $a > 1$, logo {$x - y > 1$}.
\end{enumerate}

Assim, o algoritmo funciona para $n$ composto e não quadrado perfeito.

Sabemos que, se $n$ é composto, encontraremos um valor adequado para $x$ tal que $x < \frac{n+1}{2}$. Portanto, se $x$ atingir o valor $\frac{n+1}{2}$ no decorrer do processo sem encontrar uma fatoração, significa que $n$ não é composto. Como o algoritmo é aplicado para um inteiro ímpar maior do que 1, não sendo composto, $n$ será primo. Dessa maneira, demonstramos também que o algoritmo é finito.


\subsection{Implementação Computacional}

\begin{lstlisting}[style=PythonStyle, caption=Fermat Algorithm, label=code:Fermat]
def fermat(N):
    x = math.isqrt(N)
    
    if (N%2 == 0):
        return (2, N//2)

    if (x*x == N):
        return (x, x)

    while x != (N + 1)//2:
        x = x + 1
        w = pow(x, 2) - N
        y = math.isqrt(w)
        if (y*y  == w):
            return (x-y, x+y)

    return (1, N)
\end{lstlisting}

\subsubsection{Simulação}
Como comentado no início da seção, a Fatoração de Fermat é um mecanismo útil quando os fatores do número são próximos, entretanto, vamos analisar o quão melhor o algoritmo se torna ao tomar o pior e melhor caso; isto é, no pior caso tomar um semiprimo $n = 3 \cdot p$ - tomamos 3 pois o algoritmo trata o 2 como um caso particular e por isso não segue diretamente o algoritmo -  e o melhor caso em que $p$ e $q$ são os primos mais próximos. Além disso, incrementamos os digitos do número e vemos o tempo decorrido para calcular a decomposição em primos.


\begin{table}[H]
\centering
\caption{Tabela para Fatoração de Semiprimos}
\label{tab:fatores}
\begin{tabular}{r l}
\toprule
Semiprimo (N) & Fatores (p, q) \\
\midrule
15 & 3\cdot 5\\
143 & 11\cdot 13\\
2491 & 47\cdot 53\\
47053 & 211\cdot 223\\
304679 & 547\cdot 557\\
5494327 & 2341\cdot 2347\\
76562491 & 8747\cdot 8753\\
816359183 & 28571\cdot 28573\\
7844290183 & 88547\cdot 88589\\
83274953467 & 288571\cdot 288577\\
150986644891 & 388567\cdot 388573\\
\bottomrule
\end{tabular}
\end{table}

\begin{table}[H]
\centering
\caption{Tabela para Fatoração de Semiprimos}
\label{tab:fatores_f}
\begin{tabular}{r l} 
\toprule
Semiprimo (N) & Fatores (p, q) \\
\midrule
93 & $3 \cdot 31$\\
267 & $3 \cdot 89$\\
1569 & $3 \cdot 523$\\
19473 & $3 \cdot 6491$\\
499461 & $3 \cdot 166487$\\
2899473 & $3 \cdot 966491$\\
17899473 & $3 \cdot 5966491$\\
107899431 & $3 \cdot 35966477$\\
1607899443 & $3 \cdot 535966481$\\
13607899449 & $3 \cdot 4535966483$\\
874607899119 & $3 \cdot 291535966373$\\
\bottomrule
\end{tabular}
\end{table}

\begin{figure}[H]
    \centering
    \begin{minipage}{0.49\textwidth}
        \centering
        \includegraphics[width=\linewidth]{img/nologFermat.png}
        \caption{Sem Escala Logarítmica.}
        \label{fig:a}
    \end{minipage}
    \hfill
    \begin{minipage}{0.49\textwidth}
        \centering
        \includegraphics[width=\linewidth]{img/logFermat.png}
        \caption{Escala Logarítmica.}
        \label{fig:b}
    \end{minipage}
\end{figure}

Logo, para números primos próximos a fatoração de Fermat é indiscutivelmente boa, não obstante, a eficiência decaí exponencialmente quando os primos se distam.

\subsection{Afinal, Qual é Melhor?}

Pela simulação anterior é visível a eficiência da Fatoração de Fermat para números primos próximos, entretanto, vimos que para primos distantes é um algoritmo computacionalmente ruim, então é factível indagar se ainda assim consegue ser melhor do que utilizar a força bruta. Além disso, quão melhor é para números primos próximos? 

Destarte, segue uma simulação comparando os tempos computacionais do Algoritmo de Fatoração de Fermat em relação ao Força Bruta.

Para Números Primos Próximos

\begin{figure}[H]
    \centering
    \begin{minipage}{0.49\textwidth}
        \centering
        \includegraphics[width=\linewidth]{img/compareCnolog.png}
        \caption{Sem Escala Logarítmica.}
        \label{fig:fda}
    \end{minipage}
    \hfill
    \begin{minipage}{0.49\textwidth}
        \centering
        \includegraphics[width=\linewidth]{img/compareClog.png}
        \caption{Escala Logarítmica.}
        \label{fig:fdse}
    \end{minipage}
\end{figure}

Para Números Primos Distantes

\begin{figure}[H]
    \centering
    \begin{minipage}{0.49\textwidth}
        \centering
        \includegraphics[width=\linewidth]{img/compareFnolog.png}
        \caption{Sem Escala Logarítmica.}
        \label{fig:fdac}
    \end{minipage}
    \hfill
    \begin{minipage}{0.49\textwidth}
        \centering
        \includegraphics[width=\linewidth]{img/compareFlog.png}
        \caption{Escala Logarítmica.}
        \label{fig:fdsec}
    \end{minipage}
\end{figure}

Por análise visual, vemos que para valores primos próximos - como esperado - o algoritmo de Fermat é superior; em contra partida, para primos distantes apesar da força bruta ser um algoritmo ineficiente ainda assim, consegue superar o de Fermat. Em suma, o Algoritmo de Fatoração de Fermat não é um método perfeito, isto é, não podemos/devemos utilizá-lo em qualquer situação indiscriminadamente, é necessário que haja um conhecimento prévio de que o RSA tem uma falha estrutural ao ter números da chave pública com valores próximos.
    \section{Ataque do Módulo Comum}

Sabemos que, na criptografia RSA, cada usuário torna público o par $(e,n)$.
Considere o caso em que dois usuários compartilham indevidamente o 
mesmo módulo $n$. Assim, suas chaves públicas são $(e_1,n)$ e $(e_2,n)$.

Suponha que uma mesma mensagem $m$ seja enviada para ambos os usuários. Pela 
criptografia RSA, as cifras recebidas são
$$
c_1 \equiv m^{e_1} \pmod{n}
\qquad\text{e}\qquad
c_2 \equiv m^{e_2} \pmod{n}.
$$

Suponha agora que um atacante intercepte $(c_1,c_2,e_1,e_2,n)$. Mostraremos que,
se $\gcd(e_1,e_2)=1$, então é possível recuperar a mensagem $m$ \emph{sem fatorar $n$},
independentemente de seu tamanho.

\subsection{Recuperação da mensagem}

Como $\gcd(e_1,e_2)=1$, pela Identidade de Bézout existem inteiros $a$ e $b$ tais que
$$
a e_1 + b e_2 = 1.
$$
Então, 
$$
    m \equiv m^1 \equiv m^{a e_1 + b e_2} \equiv (m^{e_1})^a (m^{e_2})^b \pmod{n}
$$

Caso algum dos coeficientes $a$ ou $b$ seja negativo, utiliza-se o inverso modular 
correspondente, que pode ser obtido pelo algoritmo estendido de Euclides. Portanto,
o atacante pode então computar
\[
m \equiv c_1^{a} \cdot c_2^{b} \pmod{n},
\]
\textbf{sem necessidade de fatorar $\mathbf{n}$.}

\subsection{Conclusão}

Assim, se dois usuários compartilham o \textbf{mesmo módulo} $n$ e cifram a mesma 
mensagem com expoentes públicos \textbf{coprimos} $e_1$ e $e_2$, então qualquer atacante 
que intercepte $(c_1,c_2,e_1,e_2,n)$ pode recuperar a mensagem $m$, violando 
completamente a segurança do RSA nesse cenário.

% \subsection*{Implementação computacional}

% Podemos, de forma iterativa achar os valores de $a$ e $b$ pelo algoritmo de
% Euclides extendido.

% \begin{lstlisting}[style=PythonStyle, caption=Extended GCD, label=code:egcd]
% def extended_gcd(a, b):
%     x0, y0 = 1, 0
%     x1, y1 = 0, 1

%     while b != 0:
%         q = a // b
%         a, b = b, a % b
%         x0, x1 = x1, x0 - q * x1
%         y0, y1 = y1, y0 - q * y1

%     return a, x0, y0   # gcd, x, y
% \end{lstlisting}
\end{document}